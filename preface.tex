\chapter*{Preface}\addcontentsline{toc}{chapter}{Preface}\markboth{Preface}{Preface}
Data analysis is central to every one of the empirical sciences.
Since the sciences are built from observations, it might not be hyperbole to say that every important problem or mistake in all of the sciences, ever, is either a mistake in the interpretation of data or else a mistake in the analysis of data.
A theoretician might object: Theory is just as important as data!
I don't disagree.
Indeed, as I will discuss extensively in what follows, data analysis is driven by theoretical considerations just as much as it is driven by the detailed properties of the data and the hardware used to take the data.
That is, I don't disagree that theory is important:
I agree that it is important precisely because theory informs the interpretation and analysis of our data.

These considerations of theory can become problematic in contemporary discussions of science.
Scientists like to think of having ``facts'' that are incontrovertible.
However, such ``facts'' are usually the outcome of data analyses, and, as I will discuss, all data analyses involve investigator choices and decisions.
What is remarkable about the sciences---and data analysis in particular---is that we can come to very reliable conclusions, and make very accurate and successful predictions, even though everything we are doing has a very strong subjective component.
That shows that, oddly, data analysis is adjacent to the philosophy of science.
This book will not avoid these uncomfortable subjects; it will embrace and discuss these complexities.

Of course in different fields, the data takes\footnote{%
In this book I will attempt to treat the noun form of ``data'' as a \emph{mass noun} like ``hair'' or ``grass.''
Thus, for example, ``the data \emph{is} noisy.''
I will probably slip up at times, because I have sometimes treated the noun form of ``data'' as a standard plural, like ``dogs,'' in my past.
It may or may not be surprising to the reader that the question of whether ``data'' is singular or plural is one of the questions I get asked most frequently as a data analyst.}
very different forms.
It is very different when studying infant development, cell division, or the impact of tariffs on national productivity.
Here I consider the kinds of data that appear in the physical sciences and in engineering domains.
For the purposes of this book data will be sets or lists or rasterized images of numerical values.
Each of these numerical values will be (usually) a measurements or instrument read-out,
and each will come (usually) with some kind of quantitative uncertainty estimate or precision expectation.

This book represents an attempt to write down a substantial fraction of everything I have learned in a career of data analysis.
My field is astrophysics and cosmology, so most of my experience is with the kinds of data taken by astronomical instruments, and with the kinds of measurements that physicists and astronomers are interested in making.
