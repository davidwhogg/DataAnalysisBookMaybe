\documentclass[10pt]{book}
\usepackage[utf8]{inputenc}
\usepackage[letterpaper]{geometry}
\usepackage{graphicx}

% page layout issues
\setlength{\textwidth}{12.25cm}
\setlength{\oddsidemargin}{0.5\paperwidth}\addtolength{\oddsidemargin}{-2.54cm}\addtolength{\oddsidemargin}{-0.5\textwidth}
\setlength{\evensidemargin}{\oddsidemargin}
\setlength{\textheight}{22.86cm}
\setlength{\topmargin}{0.5\paperheight}\addtolength{\topmargin}{-\headheight}\addtolength{\topmargin}{-\headsep}\addtolength{\topmargin}{-2.54cm}\addtolength{\topmargin}{-0.5\textheight}
\renewcommand{\MakeUppercase}[1]{#1} % hacking the latex margin defaults; this is a bad hack

% text typesetting and font issues
\renewcommand{\paragraph}[1]{\bigskip\noindent\textbf{#1}~---}
\renewcommand{\subparagraph}[1]{\medskip\noindent\textsl{#1}~---}
\newcommand{\foreign}[1]{{\slshape #1}}
\sloppy\sloppypar\raggedbottom\frenchspacing

% text macros
\newcommand{\chapterref}[1]{\chaptername~\ref{#1}}

% math macros
\newcommand{\given}{\,|\,}

\title{\bfseries%
The Theory and Practice of Data Analysis
for the Physical Sciences}
\author{David W. Hogg}
\date{July 2025}

\begin{document}

\maketitle
\tableofcontents

\chapter*{Preface}\addcontentsline{toc}{chapter}{Preface}\markboth{Preface}{Preface}
Data analysis is central to every one of the empirical sciences.
Since the sciences are built from observations, it might not be hyperbole to say that every important problem or mistake in all of the sciences, ever, is either a mistake in the interpretation of data or else a mistake in the analysis of data.
A theoretician might object: Theory is just as important as data!
I don't disagree.
Indeed, as I will discuss extensively in what follows, data analysis is driven by theoretical considerations just as much as it is driven by the detailed properties of the data and the hardware used to take the data.
That is, I don't disagree that theory is important:
I agree that it is important precisely because theory informs the interpretation and analysis of our data.

These considerations of theory can become problematic in contemporary discussions of science.
Scientists like to think of having ``facts'' that are incontrovertible.
However, such ``facts'' are usually the outcome of data analyses, and, as I will discuss, all data analyses involve investigator choices and decisions.
What is remarkable about the sciences---and data analysis in particular---is that we can come to very reliable conclusions, and make very accurate and successful predictions, even though everything we are doing has a very strong subjective component.
That shows that, oddly, data analysis is adjacent to the philosophy of science.
This book will not avoid these uncomfortable subjects; it will embrace and discuss these complexities.

Of course in different fields, the data takes\footnote{%
In this book I will attempt to treat the noun form of ``data'' as a \emph{mass noun} like ``hair'' or ``grass.''
Thus, for example, ``the data \emph{is} noisy.''
I will probably slip up at times, because I have sometimes treated the noun form of ``data'' as a standard plural, like ``dogs,'' in my past.
It may or may not be surprising to the reader that the question of whether ``data'' is singular or plural is one of the questions I get asked most frequently as a data analyst.}
very different forms.
It is very different when studying infant development, cell division, or the impact of tariffs on national productivity.
Here I consider the kinds of data that appear in the physical sciences and in engineering domains.
For the purposes of this book data will be sets or lists or rasterized images of numerical values.
Each of these numerical values will be (usually) a measurements or instrument read-out,
and each will come (usually) with some kind of quantitative uncertainty estimate or precision expectation.

This book represents an attempt to write down a substantial fraction of everything I have learned in a career of data analysis.
My field is astrophysics and cosmology, so most of my experience is with the kinds of data taken by astronomical instruments, and with the kinds of measurements that physicists and astronomers are interested in making.


\chapter*{Subjects that need a home}\addcontentsline{toc}{chapter}{Subjects that need a home}\markboth{Subjects that need a home}{Subjects that need a home}
This placeholder chapter is a place to put subjects that ought to go somewhere in the book, but I don't know where.
As I figure out where, I will move these ideas to those places.
\begin{itemize}
  \item The significance of your data is not itself significant.
  Consequence: Don't cut on SNR or significance; cut on first-order properties.
  Related: You can't know second-order things about your data as well as you know first-order things.
  Cut on first-order things in such a way that you mimic the SNR cut you have in mind, instead.
  \item How about upper limits vs detections?
  Related to forced photometry and catalog matching.
  \item Catalog matching? Is this too astronomical?
  \item \textsl{[Add more things above here.]}
\end{itemize}

\chapter{How to structure a data analysis project}

\part{Basics}

\chapter{Fitting a model to data}

% perhaps look at various papers and overleafs like the 'poisson and gaussian LF' document

\chapter{Constructing likelihood functions}\label{ch:likelihood}
\paragraph{Chapter abstract}

\section{Introduction}

\section{Expectations and noise}

\section{Standard likelihood forms}
\subsection{Gaussian likelihood function}

\subsection{Binned Poisson likelihood function}

\subsection{Binomial likelihood function}

\subsection{Unbinned, variable-rate Poisson likelihood function}

\subsection{Gaussian-process likelihood function}
Punt for later?

\section{Outliers}

\section{Discussion}


% to-do for this chapter
% ----------------------
% - write it

\chapter{Making a measurement}\label{ch:measurement}
\paragraph{Chapter abstract}

\section{Introduction}

\section{Things that feel like measurements, but aren't}

\section{The likelihood principle}

\section{Nuisance parameters}

\section{Conservative claims of measurement}

\section{Making measurements that other projects can use}

\section{Discussion}


\chapter{Estimating uncertainty}

\chapter{Model selection or decision making}

\chapter{Predicting and betting}

\chapter{Describing your model to others}\label{ch:notation}

\part{Theory}

\chapter{Probability theory for data analysis}

\chapter{Information theory}

\chapter{Should a project be frequentist or Bayesian?}
For insertion somewhere in this \chaptername.
\begin{itemize}
\item The idea that a prior can be because you think that the variable is actually, in the world, distributed like the prior.
Or it can be because you have ignorance about the variable, and that ignorance has a probabilistic form.
For example, angles can be uniformly distributed because a system is angle-mixed, or because \emph{you are not observing it at any special time}.
\item The idea that the prior on a nuisance parameter can be literally \emph{part of your model}. for example, when a vector is isotropically distributed (like an angular momentum vector for planetary systems), then you know the distribution of this parameter and you can integrate it out, even if you are a frequentist.
It becomes like a noise source.
\end{itemize}

\chapter{Choosing and testing prior pdfs}

\chapter{Causal structure and causal inference}

\chapter{Very small samples and \foreign{a posteriori} statistics}

\chapter{Exact symmetries}\label{ch:symmetry}

\chapter{Likelihood-free inference}

\part{Practice}

\chapter{Instrument calibration}

\chapter{Selection functions and censored data}

\chapter{Interpolation and very flexible models}\label{ch:flexible}

\chapter{Gaussian processes}

\chapter{Markov Chain Monte Carlo}

\chapter{Regression, including machine-learning regressions}

\chapter{Missing and heteroskedastic data}

\chapter{Robust methods and outliers}

\chapter{Model checking and visualization}

\chapter{Writing a scientific paper}

\appendix
\part*{Appendices}\addcontentsline{toc}{part}{Appendices}\markboth{}{}

\chapter{Useful probability distributions}

\chapter{Properties of Gaussian distributions}

\chapter{Data management}

\chapter{Numerical precision and code practices; testing}

\end{document}
